\documentclass[11pt]{res} % default is 10 pt
%\usepackage{helvetica} % uses helvetica postscript font (download helvetica.sty)
%\usepackage{newcent}   % uses new century schoolbook postscript font 
\setlength{\textheight}{9.5in} % increase text height to fit resume on 1 page
\newcommand\tab[1][1cm]{\hspace*{#1}}
\usepackage[utf8]{inputenc}
\usepackage{enumitem}
\usepackage{hyperref}
\usepackage{xcolor}
\hypersetup{
	colorlinks,
	linkcolor={red!50!black},
	citecolor={red!50!black},
	urlcolor={blue!50!black}
}

\usepackage{kotex}
\usepackage{amssymb}
\usepackage{soul}
\usepackage[sc]{mathpazo}
\renewcommand{\labelitemiv}{$\circ$}
\pagestyle{plain}
\setcounter{page}{1}
\pagenumbering{arabic}
\linespread{1.05}         % Palladio needs more leading (space between lines)

\newsectionwidth{0pt}  % So the text is not indented under section headings

\begin{document} 

\name{\huge \bf 박 상 훈 (朴 詳 勳)\\[12pt]} % the \\[12pt] adds a blank line after name
\address{University of South Carolina \\ 
	\#305 Gambrell Hall \\ 817 Henderson Street \\ Columbia, SC 29208}
\address{이메일: sp23@email.sc.edu \\
	\href{https://shpark.netlify.com/}{https://shparkko.netlify.com/} \\ 개인연락처: (803) 626-2252\\ \textit{ORCID ID}: \href{https://orcid.org/0000-0001-5365-0013}{0000-0001-5365-0013}}
 
                                             
\begin{resume}
                               
\section{학 력} 
\noindent {\bf 박사과정 \href{https://www.sc.edu/study/colleges_schools/artsandsciences/political_science/index.php}{미국 사우스캐롤라이나 주립대학교}} \\
 정치학과, 2018년 8월 - 현재
\begin{itemize}
 	\item[]  \hspace*{-8mm} \underline{\smash{세부전공: 비교정치, 국제관계}}
\end{itemize}
\noindent {\bf 석사, \href{http://hufspol.hufs.ac.kr/}{한국외국어대학교}} \\
정치외교학과, 2017년 2월 졸업
\begin{itemize} \itemsep -2pt 
	\item[] \hspace*{-8mm} 석사학위논문: ``독재국가와 재분배: 독재국가 유형에 따른 재분배의 경험적 분석."
	\item[] \hspace*{-5mm} 지도교수: 이재묵
	\item[] \hspace*{-5mm} 논문심사위원회: 우병원, 김웅진, 그리고 이대진
 	\item[] \hspace*{-8mm} \underline{\smash{세부전공: 비교정치, 정치학 방법론, 국제관계}}
\end{itemize}

\noindent {\bf 학사, \href{http://hufspol.hufs.ac.kr/}{한국외국어대학교}} \\
정치외교학과, 2015년 2월 졸업
\begin{itemize} \itemsep -2pt 
	\item[] \hspace*{-8mm} 주전공: 정치외교학
	\item[] \hspace*{-8mm} 부전공: 국제통상학
\end{itemize}

\section{논문 (학술지)} 
\begin{enumerate}[leftmargin=*]
	\item[7.] \textbf{박상훈}, 이재훈, 김일기. 2020. ``\href{http://www.earticle.net/Article/A378721}{전염병 관리와 신안보 전략: 아프리카 돼지열병(African Swine Fever) 사례를 중심으로 }" $\ulcorner$평화학연구$\lrcorner$ 21(2): 143-174.
	\item[6.] \textbf{박상훈}, 허재영. 2020. ``\href{journal.kstudy.com/ISS_DownLoad.asp?a_code=0g000726.pdf&inst=3345&isDownLoad=1&code=YqldZWtoSqVtJTNEMnMoNSUmN/B Z xLJTNEVHJpZQ==}{여론과 대북정책은 조응하는가? 4.27 판문점 선언 전후 국민의식조사의 경험적 연구}." $\ulcorner$담론201$\lrcorner$ 23(2): 83-113.
	\item[5.] 이재훈, \textbf{박상훈}. 2020. ``\href{https://m.nec.go.kr/portal/cmm/fms/FileDown.do?atchFileId=5d0ffae936edc42f3b4e864801f86a0f4096c44da03fede418b94067e8f9d0ea&fileSn=1&bbsId=&searchYear=}{기후와 투표 참여: 한국의 미세먼지 문제가 선거에 미치는 효과}''  $\ulcorner$선거연구$\lrcorner$ 1(12): 77-98.  
	\item[4.] \textbf{박상훈} 이재묵, 이대진. 2017. \href{http://www.dbpia.co.kr/Journal/ArticleDetail/NODE07183479?TotalCount=1&Seq=1&q=%5B%EB%8F%85%EC%9E%AC%EA%B5%AD%EA%B0%80%EC%99%80%20%EC%9E%AC%EB%B6%84%EB%B0%B0%C2%A7coldb%C2%A72%C2%A751%C2%A73%5D&searchWord=%EC%A0%84%EC%B2%B4%3D%5E%24%EB%8F%85%EC%9E%AC%EA%B5%AD%EA%B0%80%EC%99%80%20%EC%9E%AC%EB%B6%84%EB%B0%B0%5E*&Multimedia=0&isIdentifyAuthor=0&Collection=0&SearchAll=%EB%8F%85%EC%9E%AC%EA%B5%AD%EA%B0%80%EC%99%80%20%EC%9E%AC%EB%B6%84%EB%B0%B0&isFullText=0&specificParam=0&SearchMethod=0&Sort=1&SortType=desc&Page=1&PageSize=20#}{``독재국가와 재분배: 독재국가 유형에 따른 재분배의 경험적 분석."}  $\ulcorner$오토피아$\lrcorner$ 32(1), 271-314.
	\item[3.] 오수진, \textbf{박상훈}, 이재묵. 2017. \href{http://www.dbpia.co.kr/Journal/ArticleDetail/NODE07131578?TotalCount=1&Seq=1&q=%5B%EC%9D%B4%EC%9E%AC%EB%AC%B5%20%EC%98%A4%EC%88%98%EC%A7%84%20%EB%B0%95%EC%83%81%ED%9B%88%20%EA%B3%84%EA%B8%89%ED%88%AC%ED%91%9C%C2%A7coldb%C2%A72%C2%A751%C2%A73%5D&searchWord=%EC%A0%84%EC%B2%B4%3D%5E%24%EC%9D%B4%EC%9E%AC%EB%AC%B5%20%EC%98%A4%EC%88%98%EC%A7%84%20%EB%B0%95%EC%83%81%ED%9B%88%20%EA%B3%84%EA%B8%89%ED%88%AC%ED%91%9C%5E*&Multimedia=0&isIdentifyAuthor=0&Collection=0&SearchAll=%EC%9D%B4%EC%9E%AC%EB%AC%B5%20%EC%98%A4%EC%88%98%EC%A7%84%20%EB%B0%95%EC%83%81%ED%9B%88%20%EA%B3%84%EA%B8%89%ED%88%AC%ED%91%9C&isFullText=0&specificParam=0&SearchMethod=0&Sort=1&SortType=desc&Page=1&PageSize=20#}{``유권자의 계급배반과 정치지식: 제20대 총선에서 나타난 투표행태를 중심으로."} $\ulcorner$한국정치학회보$\lrcorner$ 51(1), 153-180.
	\item[2.] 박영득, \textbf{박상훈}. 2016. \href{http://kiss.kstudy.com/thesis/thesis-view.asp?key=3472800}{``브렉시트 국민투표 결정요인 분석: 기술숙련도와 노동시장에서의 고용경쟁."} $\ulcorner$세계지역연구논총$\lrcorner$ 34(3), 1-31.	
	\item[1.] \textbf{박상훈}. 2015. \href{http://search.koreanstudies.net/thesis/thesis-view.asp?key=3438155}{``복지지출과 경제성장: OECD 국가들의 복지지출 유형을 중심으로."} $\ulcorner$글로벌정치연구$\lrcorner$ 9(2), 81-107.	
\end{enumerate}


\section{심사 중인 연구}
\begin{enumerate}[leftmargin=*]
	\item[1.] \textbf{박상훈}, 이재훈. 2020. ``COVID-19가 다시 세운 장벽: 
	무엇이 민주주의를 병들게 하는가?"
\end{enumerate}

\section{진행 중인 연구}
\begin{enumerate}[leftmargin=*]
	\item[6.] ``너무도 다른 민주화: 엘리트 정치와 경제제재."
	\item[5.] ``왜 권위주의는 복지 프로그램을 제공하는가?"
	\item[4.] ``권위주의 체제와 경제성장: 왜 어떤 독재국가는 성공하는가?."
	\item[3.] ``민주주의가 무너지는 때: 불평등과 민주주의의 공고화."
	\item[2.] ``아시아의 노동없는 복지국가: 권력자원이론의 재조명."
	\item[1.] ``노동조건과 사회: 불안정하고 불모의 사회에서 노동보호의 조건적 효과."
\end{enumerate}

\section{CONFERENCE PAPERS}
\begin{itemize}
	\item[2020] ``Why Do Authoritarian Regimes Provide Welfare?'' The Annual Conference for Midwest Political Science Association, Chicago, Illinois, United States.
	\item[2020] ``When Does Water Bring Conflicts?'' Carolinas Conflict Consortium, Charlotte, North Carolina, United States. \textit{Poster Presentation}
	\item[2020] ``Why Do Authoritarian Regimes Provide Welfare?'' The Annual Conference for Southern Political Science Association, Puerto Rico, United States.
	\item[2019] ``Why Do Authoritarian Regimes Provide Welfare?'' The World Congress for Korean Politics and Society, Seoul, Korea.
	\item[2018] ``권위주의 복지국가와 불평등." 한국정치학회 하계학의회의, 부산, 한국.
	\item[2017] ``정치체제와 경제성장: 통치제도의 자율성과 제약을 중심으로." 한국정치학회 연례학술회의, 서울, 한국.
	\item[2017] ``Attitudes toward Security issues and Class Betrayal Voting." The Summer Meeting of the Korean Association of Party Studies, Busan, Korea (with Jaemook Lee and Youngdeuk Park).
	\item[2017] ``Asian Political Regimes and Economic Growth: the Non-linear Relationships and Indirect Path." The Closing Presentation of Institutes for Political Methodology, Taipei, Taiwan.
	\item[2017] ``정부부채와 복지지출: 지출의 이질적 효과를 중심으로." 한국국제정치학회 하계학술회의, 전주, 한국 (공저: 박영득).
	\item[2017] ``Attitudes to North Korea and Vote Choice: The heterogeneous effect of political knowledge of North Korea." The World Congress for Korean Politics and Society, Seoul, Korea (with Gidong Kim).
	\item[2017] ``Aid and Redistribution: the distributional effects of aid in recipient countries." The World Congress for Korean Politics and Society, Seoul, Korea (with Suna Jeong).
	\item[2016] ``제20대 총선에서 나타난 계급배반 투표와 정치지식." 한국정치학회 연례학술회의, 서울, 한국 (공저: 오수진).
	\item[2016] ``독재국가로서 북한: 무엇을 위해, 어떻게 재분배하는가." 한국평화연구학회 연례학술회의, 제주, 한국.
	\item[2016] ``독재국가와 재분배: 독재국가의 세부 유형에 따른 재분배의 경험적 분석." 한국정치학회 하계학술회의, 서울, 한국.
\end{itemize}

\section{연구 관심분야}
비교정치; 권위주의 정치체제; 민주주의 정치체제; 민주화; 체제 붕괴; 비교제도론; 불평등과 재분배의 정치경제학

\section{EXPERIENCE}
\begin{itemize}[leftmargin=*]
	\item[] \textbf{강의 관련: 조교 활동}
	\begin{itemize}[leftmargin=*]
		\item 미국 사우스캐롤라이나 주립대학교 정치학과
		\begin{itemize}[leftmargin=*]
			\setlength\itemsep{-0.2em}
			\item[] \textit{Global Politics}, 2020
			\item[] \textit{Genocide: A Comparative Perspective}, 2020
			\item[] \textit{International Relations}, 2019
			\item[] \textit{Democratic Theory}, 2019
			\item[] \textit{Comparative Politics}, 2018
		\end{itemize}
		
		\item 한국외국어대학교 정치학과
		\begin{itemize}[leftmargin=*]
			\setlength\itemsep{-0.2em}
			\item[] \textit{정치학연구방법론}, 2016
			\item[] \textit{시민사회론}, 2015-2016
		\end{itemize}
	\end{itemize}
	
	\item[] \textbf{연구 프로젝트}
	 \begin{itemize}[leftmargin=*]
	 	\item[2020] 연구책임, 공동연구원: 이재훈, 한국표준협회: ``COVID-19가 다시 세운 장벽: 무엇이 민주주의를 병들게 하는가?" (민주화운동기념사업회)
	 	\item[2016] 연구조교, 책임연구원: 이재묵, 한국외국어대학교 정치외교학과 조교수: ``국민의 당에 대한 유권자의 인식과 선호에 대한 분석." (국민의당)
	 	\item[2015] 연구조교, 책임연구원: 김일기, 국가안보전략연구원 연구원: ``비핵화 검증절차 연구." (대한민국 통일부)
	 	\item[2015] 연구조교, 책임연구원: 이상환, 한국외국어대학교 정치외교학과 교수: ``G20국가들 간 보건협력 및 갈등에 관한 연구: 21세기 국제사회의 전염병 사례를 중심으로." (한국연구재단)
	 \end{itemize}
	 
	\section{주요 수상실적} 
	\begin{itemize}[wide = 0pt] \itemsep -2pt 
		\item[2018-2023] \textbf{University of South Carolina, Department of Political Science, Columbia, SC}
		\begin{itemize} \itemsep -2pt 
			\item[$\circ$] ACAF 4.00 Graduate Assistantships
		\end{itemize} \itemsep -2pt 
		\item[2020] \textbf{Empirical Implications of Theoretical Models Institute}
		\begin{itemize} \itemsep -2pt 
			\item[$\circ$] \href{https://52.2.147.143/icpsrweb/content/sumprog/scholarships/winners-2020.html}{EITM Certification Scholarships: ICPSR 여름 프로그램 2개 세션 참가 지원} (\$4,000)
		\end{itemize} 
		\item[2020] \textbf{Department of Political Science, University of South Carolina}
		\begin{itemize} \itemsep -2pt 
			\item[$\circ$] Outstanding Graduate Student Awards
		\end{itemize} 
		\item[2020] \textbf{Department of Political Science, University of South Carolina}
		\begin{itemize} \itemsep -2pt 
			\item[$\circ$] The Second Annual Conference Funding Competition: Joint award (\$500)
		\end{itemize} 
		\item[2019] \textbf{국가안보전략연구원}
		\begin{itemize} \itemsep -2pt 
			\item[$\circ$] 제1회 신안보연구 논문대회회: 우수상 (\$2,000)
			\item[$\circ$] 제목: 전염병 관리와 신안보 전략: 한국의 아프리카 돼지열병 사례를 중심으로
		\end{itemize}
		\item[2015-2017] \textbf{한국외국어대학교}
		\begin{itemize} \itemsep -2pt 
			\item[$\circ$] HUFSAN 장학금
		\end{itemize}
	\end{itemize}
\end{itemize}
\section{학회 소속} 
 \begin{itemize}[wide = 0pt] \itemsep -2pt 
 	\item[ ] Member of Midwest Political Science Association
 	\item[ ] Member of Southern Political Science Association
\end{itemize}

\section{정규교과 외 교육과정} 
\begin{itemize}[wide = 0pt] \itemsep -2pt 
	\item[2020] 사회연구의 정량방법 여름 프로그램, 미시간대학교 (미국)
	\begin{itemize}
		\item[$\circ$] \textit{제1세션: 2020년 6월 22일-7월 17일}
		\begin{enumerate}
			\item 사회과학 연구를 위한 베이지안 모델링 I: 기본과 적용
			\item 최대가능도추정법 I: 일반화 선형모델
			\item PYTHON 입문 (7월 6일-17일)
		\end{enumerate}
		\item[$\circ$] \textit{제2세션: 2020년 7월 20일-8월 14일}
		\begin{enumerate}
			\item 게임이론
			\item 사회과학 연구를 위한 베이지안 모델링 II: 심화 주제
			\item 사회과학을 위한 인과추론
		\end{enumerate}
	\end{itemize}
	\item[2018] 사회과학데이터혁신센터,연세대학교 (한국)
	\begin{itemize}
	\item[$\circ$] R 기초강의
	\end{itemize}
	\item[2017] 정치학방법론연구회 (대만): 교차수준 분석 통계
	\begin{itemize} \itemsep -2pt 
		\item[$\circ$] 위계 연령-기간-코호트 분석
		\item[$\circ$] 다수준 분석
		\item[$\circ$] 다수준 종합분석: 공간적 미시모의실험 접근법
	\end{itemize}
	\item[2017] 일반대학원, 한국외국어대학교 (한국)
	\begin{itemize} \itemsep -2pt 
		\item[$\circ$] PYTHON을 이용한 프로그래밍 강의(자료 관리)
	\end{itemize}		
	\item[2017] 사회과학데이터혁신센터,연세대학교 (한국)
	\begin{itemize}
		\item[$\circ$]  PYTHON을 이용한 텍스트 및 네트워크 분석
	\end{itemize}
\end{itemize}

\section{전문성}
\begin{itemize}[wide=0pt]
	\item[\textit{\bf 언어}] 한국어: Native. 일본어: Native. 영어: Advanced.
	\item[\textit{\bf 프로그램}] R, STATA, Python, \LaTeX, Markdown
\end{itemize}
\mbox{}
\vfill
\centering \textit{최종 업데이트: \today}

\end{resume}
\end{document}











